\documentclass[dissertation]{uathesis}
\usepackage[]{graphicx}\usepackage[]{color}
%% maxwidth is the original width if it is less than linewidth
%% otherwise use linewidth (to make sure the graphics do not exceed the margin)
\makeatletter
\def\maxwidth{ %
  \ifdim\Gin@nat@width>\linewidth
    \linewidth
  \else
    \Gin@nat@width
  \fi
}
\makeatother

\definecolor{fgcolor}{rgb}{0.345, 0.345, 0.345}
\newcommand{\hlnum}[1]{\textcolor[rgb]{0.686,0.059,0.569}{#1}}%
\newcommand{\hlstr}[1]{\textcolor[rgb]{0.192,0.494,0.8}{#1}}%
\newcommand{\hlcom}[1]{\textcolor[rgb]{0.678,0.584,0.686}{\textit{#1}}}%
\newcommand{\hlopt}[1]{\textcolor[rgb]{0,0,0}{#1}}%
\newcommand{\hlstd}[1]{\textcolor[rgb]{0.345,0.345,0.345}{#1}}%
\newcommand{\hlkwa}[1]{\textcolor[rgb]{0.161,0.373,0.58}{\textbf{#1}}}%
\newcommand{\hlkwb}[1]{\textcolor[rgb]{0.69,0.353,0.396}{#1}}%
\newcommand{\hlkwc}[1]{\textcolor[rgb]{0.333,0.667,0.333}{#1}}%
\newcommand{\hlkwd}[1]{\textcolor[rgb]{0.737,0.353,0.396}{\textbf{#1}}}%
\let\hlipl\hlkwb

\usepackage{framed}
\makeatletter
\newenvironment{kframe}{%
 \def\at@end@of@kframe{}%
 \ifinner\ifhmode%
  \def\at@end@of@kframe{\end{minipage}}%
  \begin{minipage}{\columnwidth}%
 \fi\fi%
 \def\FrameCommand##1{\hskip\@totalleftmargin \hskip-\fboxsep
 \colorbox{shadecolor}{##1}\hskip-\fboxsep
     % There is no \\@totalrightmargin, so:
     \hskip-\linewidth \hskip-\@totalleftmargin \hskip\columnwidth}%
 \MakeFramed {\advance\hsize-\width
   \@totalleftmargin\z@ \linewidth\hsize
   \@setminipage}}%
 {\par\unskip\endMakeFramed%
 \at@end@of@kframe}
\makeatother

\definecolor{shadecolor}{rgb}{.97, .97, .97}
\definecolor{messagecolor}{rgb}{0, 0, 0}
\definecolor{warningcolor}{rgb}{1, 0, 1}
\definecolor{errorcolor}{rgb}{1, 0, 0}
\newenvironment{knitrout}{}{} % an empty environment to be redefined in TeX

\usepackage{alltt}
\newcommand{\SweaveOpts}[1]{}  % do not interfere with LaTeX
\newcommand{\SweaveInput}[1]{} % because they are not real TeX commands
\newcommand{\Sexpr}[1]{}       % will only be parsed by R


%\documentclass[dissertation,copyright]{uathesis}
%\documentclass[dissertation,CC-BY]{uathesis}
%\documentclass[dissertation,CC-BY-SA]{uathesis}
%documentclass[dissertation,CC-BY-ND]{uathesis}
%\documentclass[thesis]{uathesis}
%\documentclass[document]{uathesis}

% Package Usage
% These are the packages that we need
\usepackage{booktabs}
\usepackage{graphicx}
\usepackage{natbib}			% natbib is available on most systems, and is
					% terribly handy.
					% If you want to use a different Bibliography package, 
					% you should be able to, just change this
					% and the \bibliographystyle command below.  Be warned
					% that you may need to do a little hacking to get
					% the REFERENCES item to show up in your TOC.

% Compatibility with the AASTEX package 
% of the American Astronomical Society.
%\usepackage{deluxetable}		% Allows use of AASTEX deluxe tables
%\usepackage{aastex_hack}		% Allows other AASTEX functionality.

% These are other packages that you might find useful.
% For controlling the fonts, see
% http://www.math.uiuc.edu/~hartke/computer/latex/survey/survey.html
% The following is a nice font set:
%\usepackage{mathtime}			% Times for letters; Belleek math.
%
%\usepackage{amsmath}			% AMS Math (advanced math typesetting)
%\usepackage{lscape}			% Used for making fitting large tables in by putting them landscape
%\usepackage{refs}			
%
% If you are using hyper-ref (recommended), this command must go after all 
% other package inclusions (from the hyperref package documentation).
% The purpose of hyperref is to make the PDF created extensively
% cross-referenced.

%Also works! Change dvips to driverfallback=dvips.
\usepackage[driverfallback=dvips,bookmarks,colorlinks=true,urlcolor=black,linkcolor=black,citecolor=black]{hyperref}

%Works!
%\usepackage[pdftex,bookmarks,colorlinks=true,urlcolor=black,linkcolor=black,citecolor=black]{hyperref}
%HERE IS THE THING THAT NEEDS TO CHANGE TO GET LATEX TO WORK WITH RSTUDIO. USE pdftex instead of dvips.

% Set up some values.
\completetitle{Working Title: Perception of Patterns of Relative Formant Change in Reduced, Voiced, Stop Consonants}
\fullname{Megan Marie Willi}			% Grad college wants your full name here.
\degreename{Doctor of Philosophy}	% Title of your degree.



\begin{document}
%set_parent(‘/Users/mwilli/Documents/Spring_2017/Dissertation_Document/Dissertation_Working_Directory_Draft/Dissertation_Main.Rnw')



 



\chapter{Sample First Chapter\label{chap1}}


\section{Introduction}

I wouldn't actually name any of my chapter files chapter1.tex,
chapter2.tex, etc. End-of-sentence citation:\citep{story_2010}
   You might decide to swap them around at some
point, and then things can get messy, especially if you've labeled
things inside those files with labels like ``chap1.''  Computers
are good at counting things, let \LaTeX\ do that for you.  Be general
with your naming scheme, you can always rearrange things by altering
their order in the main dissertation.tex file.\footnote{Look at me!  I'm a footnote!}.


\section{Math Example\label{math}}

This is a real short example of using the equation environment.

\begin{equation}
	y = mx + b
\end{equation}

There is an awful lot that the equation environment and math mode
can do for you.



\includegraphics[width=\maxwidth]{figure/Exp_1_Data-1} 
% latex table generated in R 3.3.2 by xtable 1.8-2 package
% Sat Mar  4 16:29:12 2017
\begin{table}[ht]
\centering
\begin{tabular}{rllrr}
  \hline
 & Hypothesis & Vowel\_Context & Percent\_Participant\_Agreement.m & Percent\_Participant\_Agreement.s \\ 
  \hline
1 & Locus Equation & V0 & 23.03 & 4.54 \\ 
  2 & Locus Equation & V1 & 39.88 & 21.75 \\ 
  3 & Locus Equation & V2 & 23.03 & 5.72 \\ 
  4 & Locus Equation & V3 & 23.03 & 3.10 \\ 
  5 & Relative Formant Deflection Pattern & V0 & 75.27 & 6.40 \\ 
  6 & Relative Formant Deflection Pattern & V1 & 55.03 & 18.62 \\ 
  7 & Relative Formant Deflection Pattern & V2 & 73.94 & 6.15 \\ 
  8 & Relative Formant Deflection Pattern & V3 & 64.97 & 8.54 \\ 
   \hline
\end{tabular}
\caption{Demographic information} 
\end{table}


\bibliographystyle{newapa}
\bibliography{Diss_bib}
\end{document}
