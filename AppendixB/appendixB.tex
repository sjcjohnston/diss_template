% %set_parent(‘/Users/mwilli/Documents/Spring_2017/Dissertation_Document/Dissertation_Working_Directory_Draft/Dissertation_Main.Rnw')
% 
% <<chunk_options, echo=FALSE>>=
% # This is where we set basic knitr options.
% opts_chunk$set(echo=TRUE, message=FALSE, warning = FALSE, cache = FALSE, comment="")
% options(width=75) # This sets how wide the R printout can be.
% @
% 
%  <<setup-child, include=FALSE>>=
%  set_parent('/home/sam/Dissertation/Dissertation_Template/Dissertation_Main_Template/Dissertation_Main.Rnw')
% @
% 
% <<load_libraries, echo=FALSE>>=
% library(tidyr)
% library(dplyr)
% library(ggplot2)
% library(lme4)
% library(lsmeans)
% library(car)
% library(pbkrtest)
% library(xtable)
% library(cowplot)
% library(plyr)
% @

\chapter*{Appendix B}\label{appendixB}


\section{Overview}

This Appendix contains some code used for calculations described in the main report text.  Section \ref{matlab-code} contains code used for calculating the SNR of different speech audio (cf. Section \ref{chap2:observations}).  Section \ref{python-code} contains code used for calculating the WER for the human speech perception experiment results (cf. Section \ref{chap3:results}).

\section{Matlab Code for SNR Calculation}\label{matlab-code}

\subsection{Find a Signal's SNR - Main Function}

\begin{verbatim}
function [speech_file_path, SNRdB] = FindSignalSNR( speech_file_path, noise_file_path )
%  Calculate the SNR for a noisy speech file via comparing RMS values
%  between the aligned speech and noise signals
%  S. Johnston
%  04.06.17
%  Input:
%      speech_file_path = path to the audio file of speech
%      noise_file_path = path to the noise file containing the background noise in the speech file
%  Output:
%      speech_file_path = path to the audio file of speech for which SNR was calculated
%      SNRdB = SNR of the current speech audio in decibels

[speech_file,Fs] = audioread(speech_file_path);
[noise_file,~] = audioread(noise_file_path);

% Align and fit noise to given file
aligned_noise_file = AlignNoise(speech_file,noise_file);
SNRdB = CalculateSNR(speech_file, aligned_noise_file, Fs);
\end{verbatim}


\subsection{Time-Alignment of Audio}

\begin{verbatim}
function [ aligned_noise_vector ] = AlignNoise( speech_vector, noise_vector )
%  Aligns and crops the noise vector to match the noise in the speech sound file
%  S. Johnston
%  04.06.17
%  Input:
%      speech_vector = vector of speech audio samples
%      noise_vector = vector of noise audio samples
%  Output:
%      aligned_noise_vector = noise_vector that has been shifted and sliced to align with the speech vector

[C, lag] = xcorr(noise_vector, speech_vector);
[~,I] = max(abs(C));
sampleDiff = abs(lag(I));

aligned_noise_vector = noise_vector(sampleDiff:(sampleDiff + length(speech_vector)));
\end{verbatim}


\subsection{SNR Calculation}

\begin{verbatim}
function [ SNRdB ] = CalculateSNR( speech_vector, noise_vector, Fs )
%   Calculate the SNR for a given pair of sounds, speech and noise
%   Returns a rounded integer for SNR
%   S. Johnston
%   04.06.17
%   Input: 
%      speech_vector = aligned speech sample vector, 
%      noise_vector = aligned noise vector, 
%      Fs = signal sampling frequency
%  Output:
%      SNRdB = SNR in decibels

dt = 0.01;

% Calculate Hilbert Envelope
[~, ~, env] = HilbertEnvelope(speech_vector, Fs);
threshold = 0.025;
env_thresh = env-threshold;
env_thresh(env_thresh<0) = 0;
[i,j] = find(env1(:)>0);

% Only use the part of the speech vector pertaining to the Hilbert Envelope
speech_vector = speech_vector(i);

% Find RMS values
[noise_rms,~] = FindRMS(noise_vector, Fs, dt);
[speech_rms,~] = FindRMS(speech_vector, Fs, dt);

% Calculate SNR
SNR = ( mean(speech_rms) / mean(noise_rms) )^2;

% Convert to dB
SNRdB = 10*log10(SNR);
\end{verbatim}

\subsection{Hanning Window Calculation}

\begin{verbatim}
function [env,env2,env3] = HilbertEnvelope(sig,fs);
%HilbertEnvelope.m
%10.21.16
%Author: B. Story
%Calculates the Hilbert amplitude envelope of a signal
%input: sig = single channel signal, fs = sampling freq of signal

\%---calulate coeffs for butterworth envelope filters ----
[b,a] = butter(5,25/(fs/2));
[d,c] = butter(3,8/(fs/2));

hsig = hilbert(sig);
env = sqrt(real(hsig).^2 + imag(hsig).^2);

env2 = (filtfilt(b,a,env));
env3 = (filtfilt(d,c,env2));
\end{verbatim}


\subsection{RMS Calculation}

\begin{verbatim}
function [rms,t] = FindRMS(sig,fs,dt)
%FindRMS.m
%04.03.17
%B. Story, U. Arizona
%calculates rms values of signal in time period defined by dt
%INPUTS
%sig = signal
%fs = sampling freq
%dt = time window over which to calculate RMS values; dt = 0.01 is
%recommended
%OUTPUTS
%rms = vector of rms values
%t = time vector for plotting (e.g., plot(t,rms) )


N = round(dt*fs);
dtn = N/fs;
rms = [];

j = 1;
for i = 1:N:length(sig)-N
  x = sig(i:(i-1)+N);
  x = x .^ 2;
  x = mean(x);
  x = sqrt(x);
      rms(j) = x;
    t(j) = (dtn/2) + (j-1)*dtn;
  j = j+1;
end;
\end{verbatim}


\section{Python Code for WER Calculation}\label{python-code}

This code was borrowed from http://progfruits.blogspot.com/2014/02/word-error-rate-wer-and-word.html on April 28th, 2017.  This script was used in the calculation of word error rate (WER) for the human perception study (cf. Chapter \ref{chapter3}).  The WER calculation for the ASR studies in Chapter \ref{chapter4} was performed with Kaldi's built-in scoring algorithms.

\begin{verbatim}
def wer(ref, hyp):

  # Reference sentence and hypothesis sentence
  r = ref.split()
  h = hyp.split()

  DEL_PENALTY = 1
  INS_PENALTY = 1
  SUB_PENALTY = 1

  #costs will holds the costs, like in the Levenshtein distance algorithm
  costs = [[0 for inner in range(len(h)+1)] for outer in range(len(r)+1)]
  # backtrace will hold the operations we've done.
  # so we could later backtrace, like the WER algorithm requires us to.
  backtrace = [[0 for inner in range(len(h)+1)] for outer in range(len(r)+1)]

  OP_OK = 0
  OP_SUB = 1
  OP_INS = 2
  OP_DEL = 3
   
  # First column represents the case where we achieve zero
  # hypothesis words by deleting all reference words.
  for i in range(1, len(r)+1):
    costs[i][0] = DEL_PENALTY*i
    backtrace[i][0] = OP_DEL
     
  # First row represents the case where we achieve the hypothesis
  # by inserting all hypothesis words into a zero-length reference.
  for j in range(1, len(h) + 1):
    costs[0][j] = INS_PENALTY * j
    backtrace[0][j] = OP_INS
   
  # computation
  for i in range(1, len(r)+1):
    for j in range(1, len(h)+1):
      if r[i-1] == h[j-1]:
        costs[i][j] = costs[i-1][j-1]
        backtrace[i][j] = OP_OK
      else:
        substitutionCost = costs[i-1][j-1] + SUB_PENALTY # penalty is always 1
        insertionCost   = costs[i][j-1] + INS_PENALTY   # penalty is always 1
        deletionCost   = costs[i-1][j] + DEL_PENALTY   # penalty is always 1
         
        costs[i][j] = min(substitutionCost, insertionCost, deletionCost)
        if costs[i][j] == substitutionCost:
          backtrace[i][j] = OP_SUB
        elif costs[i][j] == insertionCost:
          backtrace[i][j] = OP_INS
        else:
          backtrace[i][j] = OP_DEL
         
  # back trace though the best route:
  i = len(r)
  j = len(h)
  numSub = 0
  numDel = 0
  numIns = 0
  numCor = 0
  
  while i > 0 or j > 0:
    if backtrace[i][j] == OP_OK:
      numCor += 1
      i-=1
      j-=1
    elif backtrace[i][j] == OP_SUB:
      numSub +=1
      i-=1
      j-=1
    elif backtrace[i][j] == OP_INS:
      numIns += 1
      j-=1
    elif backtrace[i][j] == OP_DEL:
      numDel += 1
      i-=1

  wer_result = round( (numSub + numDel + numIns) / (float) (len(r)), 3)
  return wer_result

\end{verbatim}


